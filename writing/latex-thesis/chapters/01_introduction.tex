% !TeX root = ../main.tex
% Add the above to each chapter to make compiling the PDF easier in some editors.

\chapter{Introduction}\label{chapter:introduction}
\section{IoT \& Distributed Sensor Networks}
\subsection{Show how Iot is being currently used, its pros and cons}
\subsection{Give an idea about the devices used to make a distributed sensor network}
\section{Motivation}
\subsection{Show the need to explore Pervasive Computing}

\subsection{Illustrate why it might be better to distribute the data in some cases rather than accumulating it in a single server}

\subsection{Explain why Cloud Computing is not always the right solution in some cases}

\subsection{Explain the need to find IoT devices capabilities and limitations when used for data computation}


%It is believed that development of smart pervasive computing devices that uses sensors and actuators are mainly grouped into smaller disconnected architectures and thus hard to create a composite framework in which all devices can communicate and integrate\cite{5470524}. In order to tackle this problem and to ensure that our computational model design is dynamic and flexible enough, we must ensure that our model is composable. By Composability we mean that computational models should be able to communicate, send and receive input and output data. 
