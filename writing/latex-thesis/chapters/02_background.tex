% !TeX root = ../main.tex
% Add the above to each chapter to make compiling the PDF easier in some editors.

\chapter{Background \& Related Work}\label{chapter:background}

This chapter describes the concepts and background information that this thesis uses and relies on. It gives a brief introduction about Internet of things and other concepts that emerged from within such as pervasive and fog computing. Moreover, the chapter explains delay tolerant and information centric networking as they play an important role in this thesis. Further, we explain the software platforms and hardware used to implement the system framework.

\section{Internet of Things}

Internet of things \textbf{IoT} is one of the most trending topics in the software industry. In general terms, IoT refers to a highly dynamic and scalable distributed network of connected devices equipped with context-aware gadgets that enables them to see, hear and think\cite{DAC:DAC2417}. Then, transfer theses senses to a stream of information allowing them to digest the data and act intelligently through the actuators if needed. They are also allowed to communicate and share knowledge, which make them smart, powerful and capable of acting independently. Smart devices in an IoT network are heterogeneous in terms of computation capabilities, also each device should be energy optimized and able to communicate. Moreover, to qualify for being smart, devices must have a unique global identifier, name, address and can sense the environment. Since smart devices have unique identifier and are context-aware, they can be tracked and localized, which is very helpful when performing geospatial computations \cite{Miorandi20121497}. The huge demand on IoT has triggered the development small-scale, high-performance, low-cost computers, in addition, sensors and actuators are getting cheaper, smaller and more powerful which in turn increased the interest even more.
 

The IoT concept can be viewed from a wide angle of perspectives, it is very elastic and provides an enormous scale of opportunities in different areas. Currently the number of connected smart devices are estimated in billions, they aim to automate everything around us and are mainly targeted to increase life quality. The broad range of IoT applications include:
\begin{itemize}
\item Smart homes which tend to use sensors and actuators to monitor and optimize home resource consumption and control home devices in away that increases humans satisfaction levels. 
\item Smart factories also known as "Industry 4.0" the fourth industrial revolution which are optimized machines that communicate together in order optimize the manufacturing process and gather data to analyze factories logistics, pipeline and product availability It also creates intelligent products that can be located and identified at all times in the process \cite{Gilchrist:2016:III:2994178}.

\item Smart cities is one of the most adopted applications in the IoT field, it compromises smart parking,  pollution detection, traffic congestion monitoring and control, real time noise analysis, smart roads, waste management and others.  All this applications need enhanced communication and data infrastructure. It aims to increase quality of living for individuals.

\item There are also applications in  health care, environmental monitoring, security and surveillance.
\end{itemize}

IoT is very diverse, one way of applying it is to gather data from the smart devices, then process data in the cloud via Cloud-Computing. Afterwards, results could be sent back to smart devices in order to act somehow. However, exploiting the overwhelming capacity of IoT lead to more specialized and concrete terms that are more focused on pushing computations to the smart devices "Edges". Consequently, more terms like Edge-Computing, Pervasive-Computing and Fog-Computing emerged.


\subsection{ Wireless Sensor Networks}

\subsection{Pervasive Computing} 

\subsection{Fog Computing}



\section{Networking}
\subsection{Delay Tolerant Networking}
\subsection{Information Centric Networking}

\section{Used Platforms}
\subsection{Node-Red}
\subsection{SCAMPI}
\subsection{Raspberry Pi}
\subsection{Time-series Databases}

%\section{Illustrate what are the ideas and possible network mechanisms and protocols that could be used data transfer}
%\subsection{Server To Server }
%\subsection{Server To Device }
%\subsection{Device To Device }

